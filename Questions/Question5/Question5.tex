\documentclass[11pt,preprint, authoryear]{elsarticle}

\usepackage{lmodern}
%%%% My spacing
\usepackage{setspace}
\setstretch{1.2}
\DeclareMathSizes{12}{14}{10}{10}

% Wrap around which gives all figures included the [H] command, or places it "here". This can be tedious to code in Rmarkdown.
\usepackage{float}
\let\origfigure\figure
\let\endorigfigure\endfigure
\renewenvironment{figure}[1][2] {
    \expandafter\origfigure\expandafter[H]
} {
    \endorigfigure
}

\let\origtable\table
\let\endorigtable\endtable
\renewenvironment{table}[1][2] {
    \expandafter\origtable\expandafter[H]
} {
    \endorigtable
}


\usepackage{ifxetex,ifluatex}
\usepackage{fixltx2e} % provides \textsubscript
\ifnum 0\ifxetex 1\fi\ifluatex 1\fi=0 % if pdftex
  \usepackage[T1]{fontenc}
  \usepackage[utf8]{inputenc}
\else % if luatex or xelatex
  \ifxetex
    \usepackage{mathspec}
    \usepackage{xltxtra,xunicode}
  \else
    \usepackage{fontspec}
  \fi
  \defaultfontfeatures{Mapping=tex-text,Scale=MatchLowercase}
  \newcommand{\euro}{€}
\fi

\usepackage{amssymb, amsmath, amsthm, amsfonts}

\def\bibsection{\section*{References}} %%% Make "References" appear before bibliography


\usepackage[round]{natbib}

\usepackage{longtable}
\usepackage[margin=2.3cm,bottom=2cm,top=2.5cm, includefoot]{geometry}
\usepackage{fancyhdr}
\usepackage[bottom, hang, flushmargin]{footmisc}
\usepackage{graphicx}
\numberwithin{equation}{section}
\numberwithin{figure}{section}
\numberwithin{table}{section}
\setlength{\parindent}{0cm}
\setlength{\parskip}{1.3ex plus 0.5ex minus 0.3ex}
\usepackage{textcomp}
\renewcommand{\headrulewidth}{0pt}

\usepackage{array}
\newcolumntype{x}[1]{>{\centering\arraybackslash\hspace{0pt}}p{#1}}

%%%%  Remove the "preprint submitted to" part. Don't worry about this either, it just looks better without it:
\makeatletter
\def\ps@pprintTitle{%
  \let\@oddhead\@empty
  \let\@evenhead\@empty
  \let\@oddfoot\@empty
  \let\@evenfoot\@oddfoot
}
\makeatother

 \def\tightlist{} % This allows for subbullets!

\usepackage{hyperref}
\hypersetup{breaklinks=true,
            bookmarks=true,
            colorlinks=true,
            citecolor=blue,
            urlcolor=blue,
            linkcolor=blue,
            pdfborder={0 0 0}}


% The following packages allow huxtable to work:
\usepackage{siunitx}
\usepackage{multirow}
\usepackage{hhline}
\usepackage{calc}
\usepackage{tabularx}
\usepackage{booktabs}
\usepackage{caption}


\newenvironment{columns}[1][]{}{}

\newenvironment{column}[1]{\begin{minipage}{#1}\ignorespaces}{%
\end{minipage}
\ifhmode\unskip\fi
\aftergroup\useignorespacesandallpars}

\def\useignorespacesandallpars#1\ignorespaces\fi{%
#1\fi\ignorespacesandallpars}

\makeatletter
\def\ignorespacesandallpars{%
  \@ifnextchar\par
    {\expandafter\ignorespacesandallpars\@gobble}%
    {}%
}
\makeatother

\newenvironment{CSLReferences}[2]{%
}

\urlstyle{same}  % don't use monospace font for urls
\setlength{\parindent}{0pt}
\setlength{\parskip}{6pt plus 2pt minus 1pt}
\setlength{\emergencystretch}{3em}  % prevent overfull lines
\setcounter{secnumdepth}{5}

%%% Use protect on footnotes to avoid problems with footnotes in titles
\let\rmarkdownfootnote\footnote%
\def\footnote{\protect\rmarkdownfootnote}
\IfFileExists{upquote.sty}{\usepackage{upquote}}{}

%%% Include extra packages specified by user
\usepackage{booktabs}
\usepackage{longtable}
\usepackage{array}
\usepackage{multirow}
\usepackage{wrapfig}
\usepackage{float}
\usepackage{colortbl}
\usepackage{pdflscape}
\usepackage{tabu}
\usepackage{threeparttable}
\usepackage{threeparttablex}
\usepackage[normalem]{ulem}
\usepackage{makecell}
\usepackage{xcolor}

%%% Hard setting column skips for reports - this ensures greater consistency and control over the length settings in the document.
%% page layout
%% paragraphs
\setlength{\baselineskip}{12pt plus 0pt minus 0pt}
\setlength{\parskip}{12pt plus 0pt minus 0pt}
\setlength{\parindent}{0pt plus 0pt minus 0pt}
%% floats
\setlength{\floatsep}{12pt plus 0 pt minus 0pt}
\setlength{\textfloatsep}{20pt plus 0pt minus 0pt}
\setlength{\intextsep}{14pt plus 0pt minus 0pt}
\setlength{\dbltextfloatsep}{20pt plus 0pt minus 0pt}
\setlength{\dblfloatsep}{14pt plus 0pt minus 0pt}
%% maths
\setlength{\abovedisplayskip}{12pt plus 0pt minus 0pt}
\setlength{\belowdisplayskip}{12pt plus 0pt minus 0pt}
%% lists
\setlength{\topsep}{10pt plus 0pt minus 0pt}
\setlength{\partopsep}{3pt plus 0pt minus 0pt}
\setlength{\itemsep}{5pt plus 0pt minus 0pt}
\setlength{\labelsep}{8mm plus 0mm minus 0mm}
\setlength{\parsep}{\the\parskip}
\setlength{\listparindent}{\the\parindent}
%% verbatim
\setlength{\fboxsep}{5pt plus 0pt minus 0pt}



\begin{document}



\begin{frontmatter}  %

\title{Question 5: Volatility Modelling}

% Set to FALSE if wanting to remove title (for submission)




\author[Add1]{Ruan Geldenhuys}
\ead{22550801@sun.ac.za}





\address[Add1]{Stellenbosch University, Stellenbosch, South Africa}



\vspace{1cm}





\vspace{0.5cm}

\end{frontmatter}

\setcounter{footnote}{0}



%________________________
% Header and Footers
%%%%%%%%%%%%%%%%%%%%%%%%%%%%%%%%%
\pagestyle{fancy}
\chead{}
\rhead{}
\lfoot{}
\rfoot{}
\lhead{}
%\rfoot{\footnotesize Page \thepage } % "e.g. Page 2"
\cfoot{}

%\setlength\headheight{30pt}
%%%%%%%%%%%%%%%%%%%%%%%%%%%%%%%%%
%________________________

\headsep 35pt % So that header does not go over title




\hypertarget{introduction}{%
\section{Introduction}\label{introduction}}

This question investigates the Rand's notorious volatility. To this I
first fit a univariate GARCH in order to investigate the structural
volatility of the Rand. I then compare it with other highly volatile
currencies. Finally, I fit a multivariate GARCH between the Rand and a
G10 currency basket. It appears that, although the Rand is volatile, it
is far from the most volatile currency.

\begin{figure}[H]

{\centering \includegraphics{Question5_files/figure-latex/Figure1-1} 

}

\caption{Conditional SD of Rand}\label{fig:Figure1}
\end{figure}

The plot above shows the structural volatility of the Rand compared to
the noise.

\begin{tabular}{l|r}
\hline
Name & SD\\
\hline
Nigeria & 0.0382584\\
\hline
Romania & 0.0328795\\
\hline
Bulgaria & 0.0169528\\
\hline
Zambia & 0.0144492\\
\hline
Ghana & 0.0132018\\
\hline
Russia & 0.0123983\\
\hline
Brazil & 0.0122549\\
\hline
Turkey & 0.0121950\\
\hline
Argentina & 0.0104157\\
\hline
SouthAfrica & 0.0095090\\
\hline
\end{tabular}

The table above shows the top 10 most volatile currencies ranked. South
Africa is ranked 10th. While not the most volatile, it is certainly
high, considering we have 41 currencies in the dataset. Nigeria takes
the top spot with a SD of 0.0383.

\begin{figure}[H]

{\centering \includegraphics{Question5_files/figure-latex/Figure2-1} 

}

\caption{Conditional Standard Deviations Over Time}\label{fig:Figure2}
\end{figure}

The plot above shows the conditional standard deviations over time for
the top 10 most volatile countries. What is immediately apparent is that
every currency experience volatility spikes. However, these spikes occur
at different times for different countries, indicating that the
structural volatilities don't tend to follow each other.

\begin{verbatim}
## Sample mean of the returns:  7.9159e-05 0.0001874877 
## Component:  1 
## Estimates:  0 0.149704 0.873896 
## se.coef  :  0 0.009268 0.006297 
## t-value  :  0.016086 16.15231 138.7861 
## Component:  2 
## Estimates:  2e-06 0.069454 0.918007 
## se.coef  :  0 0.006724 0.008132 
## t-value  :  4.480019 10.32997 112.8907
\end{verbatim}

\begin{verbatim}
## Estimates:  0.95 0.03294096 7.796205 
## st.errors:  NaN NaN 0.4423782 
## t-values:   NaN NaN 17.6234
\end{verbatim}

\begin{figure}[H]

{\centering \includegraphics{Question5_files/figure-latex/Figure3-1} 

}

\caption{Dynamic Conditional Correlations: ZAR and G10}\label{fig:Figure3}
\end{figure}

The plot above shows the dynamic conditional correlation of the ZAR and
the G10 basket.

\bibliography{Tex/ref}





\end{document}
