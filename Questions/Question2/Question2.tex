\documentclass[11pt,preprint, authoryear]{elsarticle}

\usepackage{lmodern}
%%%% My spacing
\usepackage{setspace}
\setstretch{1.2}
\DeclareMathSizes{12}{14}{10}{10}

% Wrap around which gives all figures included the [H] command, or places it "here". This can be tedious to code in Rmarkdown.
\usepackage{float}
\let\origfigure\figure
\let\endorigfigure\endfigure
\renewenvironment{figure}[1][2] {
    \expandafter\origfigure\expandafter[H]
} {
    \endorigfigure
}

\let\origtable\table
\let\endorigtable\endtable
\renewenvironment{table}[1][2] {
    \expandafter\origtable\expandafter[H]
} {
    \endorigtable
}


\usepackage{ifxetex,ifluatex}
\usepackage{fixltx2e} % provides \textsubscript
\ifnum 0\ifxetex 1\fi\ifluatex 1\fi=0 % if pdftex
  \usepackage[T1]{fontenc}
  \usepackage[utf8]{inputenc}
\else % if luatex or xelatex
  \ifxetex
    \usepackage{mathspec}
    \usepackage{xltxtra,xunicode}
  \else
    \usepackage{fontspec}
  \fi
  \defaultfontfeatures{Mapping=tex-text,Scale=MatchLowercase}
  \newcommand{\euro}{€}
\fi

\usepackage{amssymb, amsmath, amsthm, amsfonts}

\def\bibsection{\section*{References}} %%% Make "References" appear before bibliography


\usepackage[round]{natbib}

\usepackage{longtable}
\usepackage[margin=2.3cm,bottom=2cm,top=2.5cm, includefoot]{geometry}
\usepackage{fancyhdr}
\usepackage[bottom, hang, flushmargin]{footmisc}
\usepackage{graphicx}
\numberwithin{equation}{section}
\numberwithin{figure}{section}
\numberwithin{table}{section}
\setlength{\parindent}{0cm}
\setlength{\parskip}{1.3ex plus 0.5ex minus 0.3ex}
\usepackage{textcomp}
\renewcommand{\headrulewidth}{0pt}

\usepackage{array}
\newcolumntype{x}[1]{>{\centering\arraybackslash\hspace{0pt}}p{#1}}

%%%%  Remove the "preprint submitted to" part. Don't worry about this either, it just looks better without it:
\makeatletter
\def\ps@pprintTitle{%
  \let\@oddhead\@empty
  \let\@evenhead\@empty
  \let\@oddfoot\@empty
  \let\@evenfoot\@oddfoot
}
\makeatother

 \def\tightlist{} % This allows for subbullets!

\usepackage{hyperref}
\hypersetup{breaklinks=true,
            bookmarks=true,
            colorlinks=true,
            citecolor=blue,
            urlcolor=blue,
            linkcolor=blue,
            pdfborder={0 0 0}}


% The following packages allow huxtable to work:
\usepackage{siunitx}
\usepackage{multirow}
\usepackage{hhline}
\usepackage{calc}
\usepackage{tabularx}
\usepackage{booktabs}
\usepackage{caption}


\newenvironment{columns}[1][]{}{}

\newenvironment{column}[1]{\begin{minipage}{#1}\ignorespaces}{%
\end{minipage}
\ifhmode\unskip\fi
\aftergroup\useignorespacesandallpars}

\def\useignorespacesandallpars#1\ignorespaces\fi{%
#1\fi\ignorespacesandallpars}

\makeatletter
\def\ignorespacesandallpars{%
  \@ifnextchar\par
    {\expandafter\ignorespacesandallpars\@gobble}%
    {}%
}
\makeatother

\newenvironment{CSLReferences}[2]{%
}

\urlstyle{same}  % don't use monospace font for urls
\setlength{\parindent}{0pt}
\setlength{\parskip}{6pt plus 2pt minus 1pt}
\setlength{\emergencystretch}{3em}  % prevent overfull lines
\setcounter{secnumdepth}{5}

%%% Use protect on footnotes to avoid problems with footnotes in titles
\let\rmarkdownfootnote\footnote%
\def\footnote{\protect\rmarkdownfootnote}
\IfFileExists{upquote.sty}{\usepackage{upquote}}{}

%%% Include extra packages specified by user

%%% Hard setting column skips for reports - this ensures greater consistency and control over the length settings in the document.
%% page layout
%% paragraphs
\setlength{\baselineskip}{12pt plus 0pt minus 0pt}
\setlength{\parskip}{12pt plus 0pt minus 0pt}
\setlength{\parindent}{0pt plus 0pt minus 0pt}
%% floats
\setlength{\floatsep}{12pt plus 0 pt minus 0pt}
\setlength{\textfloatsep}{20pt plus 0pt minus 0pt}
\setlength{\intextsep}{14pt plus 0pt minus 0pt}
\setlength{\dbltextfloatsep}{20pt plus 0pt minus 0pt}
\setlength{\dblfloatsep}{14pt plus 0pt minus 0pt}
%% maths
\setlength{\abovedisplayskip}{12pt plus 0pt minus 0pt}
\setlength{\belowdisplayskip}{12pt plus 0pt minus 0pt}
%% lists
\setlength{\topsep}{10pt plus 0pt minus 0pt}
\setlength{\partopsep}{3pt plus 0pt minus 0pt}
\setlength{\itemsep}{5pt plus 0pt minus 0pt}
\setlength{\labelsep}{8mm plus 0mm minus 0mm}
\setlength{\parsep}{\the\parskip}
\setlength{\listparindent}{\the\parindent}
%% verbatim
\setlength{\fboxsep}{5pt plus 0pt minus 0pt}



\begin{document}



\begin{frontmatter}  %

\title{Question 2: Currency Hedging Analysis}

% Set to FALSE if wanting to remove title (for submission)




\author[Add1]{Ruan Geldenhuys}
\ead{22550801@sun.ac.za}





\address[Add1]{Stellenbosch University, Stellenbosch, South Africa}



\vspace{1cm}





\vspace{0.5cm}

\end{frontmatter}

\setcounter{footnote}{0}



%________________________
% Header and Footers
%%%%%%%%%%%%%%%%%%%%%%%%%%%%%%%%%
\pagestyle{fancy}
\chead{}
\rhead{}
\lfoot{}
\rfoot{}
\lhead{}
%\rfoot{\footnotesize Page \thepage } % "e.g. Page 2"
\cfoot{}

%\setlength\headheight{30pt}
%%%%%%%%%%%%%%%%%%%%%%%%%%%%%%%%%
%________________________

\headsep 35pt % So that header does not go over title




\hypertarget{introduction}{%
\section{Introduction}\label{introduction}}

This question focuses on a recent article that discusses the
misconception surrounding hedging out the Rand. By replicating the
figure within that article I therefore show that hedging vary rarely has
beneficial effects. I then go further by showing how the volatility of a
portfolio can actually be increased through hedging

\begin{figure}[H]

{\centering \includegraphics{Question2_files/figure-latex/Figure1-1} 

}

\caption{Replicated Figure}\label{fig:Figure1}
\end{figure}

The graph interpretation can be done as follows. If we find ourselves in
the top left quadrant, it means portfolio returns are positive and
USD/ZAR returns are negative. In this case a hedge will work but it
increases volatility, a less than desirable outcome. This occurs for
46.2\% of observations and 30.1\% if we apply a fee. The bottom left
quadrant shows when portfolio and USD/ZAR returns are negative. This is
the best case scenario for a hedge, yet it only occurs 7.5\% of the
time. 2.2\% when applying a fee. The top right and bottom right quadrant
shows outright negative positions to be in when hedging against the
Rand. These occur 19.9\% and 25.8\% respectively.

\begin{figure}[H]

{\centering \includegraphics{Question2_files/figure-latex/Figure2-1} 

}

\caption{Rolling 3 Year Annualized SD}\label{fig:Figure2}
\end{figure}

The graph displayed above illustrates the rolling three-year annualized
standard deviation for both a hedged and an unhedged portfolio. What
stands out is that, for most of the observed period, the unhedged
portfolio exhibits lower volatility. This observation reinforces the
points made earlier, demonstrating that hedging against the Rand
actually elevates the portfolio's overall volatility. This outcome
contradicts the primary goal of hedging against the Rand, which is to
reduce volatility.

\bibliography{Tex/ref}





\end{document}
