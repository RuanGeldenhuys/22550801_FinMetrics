\documentclass[11pt,preprint, authoryear]{elsarticle}

\usepackage{lmodern}
%%%% My spacing
\usepackage{setspace}
\setstretch{1.2}
\DeclareMathSizes{12}{14}{10}{10}

% Wrap around which gives all figures included the [H] command, or places it "here". This can be tedious to code in Rmarkdown.
\usepackage{float}
\let\origfigure\figure
\let\endorigfigure\endfigure
\renewenvironment{figure}[1][2] {
    \expandafter\origfigure\expandafter[H]
} {
    \endorigfigure
}

\let\origtable\table
\let\endorigtable\endtable
\renewenvironment{table}[1][2] {
    \expandafter\origtable\expandafter[H]
} {
    \endorigtable
}


\usepackage{ifxetex,ifluatex}
\usepackage{fixltx2e} % provides \textsubscript
\ifnum 0\ifxetex 1\fi\ifluatex 1\fi=0 % if pdftex
  \usepackage[T1]{fontenc}
  \usepackage[utf8]{inputenc}
\else % if luatex or xelatex
  \ifxetex
    \usepackage{mathspec}
    \usepackage{xltxtra,xunicode}
  \else
    \usepackage{fontspec}
  \fi
  \defaultfontfeatures{Mapping=tex-text,Scale=MatchLowercase}
  \newcommand{\euro}{€}
\fi

\usepackage{amssymb, amsmath, amsthm, amsfonts}

\def\bibsection{\section*{References}} %%% Make "References" appear before bibliography


\usepackage[round]{natbib}

\usepackage{longtable}
\usepackage[margin=2.3cm,bottom=2cm,top=2.5cm, includefoot]{geometry}
\usepackage{fancyhdr}
\usepackage[bottom, hang, flushmargin]{footmisc}
\usepackage{graphicx}
\numberwithin{equation}{section}
\numberwithin{figure}{section}
\numberwithin{table}{section}
\setlength{\parindent}{0cm}
\setlength{\parskip}{1.3ex plus 0.5ex minus 0.3ex}
\usepackage{textcomp}
\renewcommand{\headrulewidth}{0pt}

\usepackage{array}
\newcolumntype{x}[1]{>{\centering\arraybackslash\hspace{0pt}}p{#1}}

%%%%  Remove the "preprint submitted to" part. Don't worry about this either, it just looks better without it:
\makeatletter
\def\ps@pprintTitle{%
  \let\@oddhead\@empty
  \let\@evenhead\@empty
  \let\@oddfoot\@empty
  \let\@evenfoot\@oddfoot
}
\makeatother

 \def\tightlist{} % This allows for subbullets!

\usepackage{hyperref}
\hypersetup{breaklinks=true,
            bookmarks=true,
            colorlinks=true,
            citecolor=blue,
            urlcolor=blue,
            linkcolor=blue,
            pdfborder={0 0 0}}


% The following packages allow huxtable to work:
\usepackage{siunitx}
\usepackage{multirow}
\usepackage{hhline}
\usepackage{calc}
\usepackage{tabularx}
\usepackage{booktabs}
\usepackage{caption}


\newenvironment{columns}[1][]{}{}

\newenvironment{column}[1]{\begin{minipage}{#1}\ignorespaces}{%
\end{minipage}
\ifhmode\unskip\fi
\aftergroup\useignorespacesandallpars}

\def\useignorespacesandallpars#1\ignorespaces\fi{%
#1\fi\ignorespacesandallpars}

\makeatletter
\def\ignorespacesandallpars{%
  \@ifnextchar\par
    {\expandafter\ignorespacesandallpars\@gobble}%
    {}%
}
\makeatother

\newenvironment{CSLReferences}[2]{%
}

\urlstyle{same}  % don't use monospace font for urls
\setlength{\parindent}{0pt}
\setlength{\parskip}{6pt plus 2pt minus 1pt}
\setlength{\emergencystretch}{3em}  % prevent overfull lines
\setcounter{secnumdepth}{5}

%%% Use protect on footnotes to avoid problems with footnotes in titles
\let\rmarkdownfootnote\footnote%
\def\footnote{\protect\rmarkdownfootnote}
\IfFileExists{upquote.sty}{\usepackage{upquote}}{}

%%% Include extra packages specified by user

%%% Hard setting column skips for reports - this ensures greater consistency and control over the length settings in the document.
%% page layout
%% paragraphs
\setlength{\baselineskip}{12pt plus 0pt minus 0pt}
\setlength{\parskip}{12pt plus 0pt minus 0pt}
\setlength{\parindent}{0pt plus 0pt minus 0pt}
%% floats
\setlength{\floatsep}{12pt plus 0 pt minus 0pt}
\setlength{\textfloatsep}{20pt plus 0pt minus 0pt}
\setlength{\intextsep}{14pt plus 0pt minus 0pt}
\setlength{\dbltextfloatsep}{20pt plus 0pt minus 0pt}
\setlength{\dblfloatsep}{14pt plus 0pt minus 0pt}
%% maths
\setlength{\abovedisplayskip}{12pt plus 0pt minus 0pt}
\setlength{\belowdisplayskip}{12pt plus 0pt minus 0pt}
%% lists
\setlength{\topsep}{10pt plus 0pt minus 0pt}
\setlength{\partopsep}{3pt plus 0pt minus 0pt}
\setlength{\itemsep}{5pt plus 0pt minus 0pt}
\setlength{\labelsep}{8mm plus 0mm minus 0mm}
\setlength{\parsep}{\the\parskip}
\setlength{\listparindent}{\the\parindent}
%% verbatim
\setlength{\fboxsep}{5pt plus 0pt minus 0pt}



\begin{document}



\begin{frontmatter}  %

\title{Question 1: Systematic AI Fund}

% Set to FALSE if wanting to remove title (for submission)




\author[Add1]{Ruan Geldenhuys}
\ead{22550801@sun.ac.za}





\address[Add1]{Stellenbosch University, Stellenbosch, South Africa}



\vspace{1cm}





\vspace{0.5cm}

\end{frontmatter}

\setcounter{footnote}{0}



%________________________
% Header and Footers
%%%%%%%%%%%%%%%%%%%%%%%%%%%%%%%%%
\pagestyle{fancy}
\chead{}
\rhead{}
\lfoot{}
\rfoot{}
\lhead{}
%\rfoot{\footnotesize Page \thepage } % "e.g. Page 2"
\cfoot{}

%\setlength\headheight{30pt}
%%%%%%%%%%%%%%%%%%%%%%%%%%%%%%%%%
%________________________

\headsep 35pt % So that header does not go over title




\hypertarget{question-1}{%
\section{Question 1}\label{question-1}}

This question investigates returns from an AI Fund, a market benchmark
(CAPPED SWIX) and active managers. It aims to shed light on the
differences and similarity in return structures through various
visualisation techniques. I find that the benchmark often outperforms
both the AI Fund and the active managers.

\begin{figure}[H]

{\centering \includegraphics{Question1_files/figure-latex/Figure1-1} 

}

\caption{AI vs Active Manager Returns}\label{fig:Figure1}
\end{figure}

The figure above shows a scatter between the returns of active managers
and the AI fund. A clear positive correlation is visible.

\begin{figure}[H]

{\centering \includegraphics{Question1_files/figure-latex/Figure2-1} 

}

\caption{Ai vs Benchmark Returns}\label{fig:Figure2}
\end{figure}

The figure above now shows a scatter between the returns of the
benchmark and the AI fund. An even stronger positive correlation is now
visible. This means that for, in both cases, it is rare for the AI fund
to show a negative return when the other funds show a positive return
and vice-versa.

\begin{figure}[H]

{\centering \includegraphics{Question1_files/figure-latex/Figure3-1} 

}

\caption{Rolling 3 Year Annualized Returns Comparison}\label{fig:Figure3}
\end{figure}

The rolling 3 year annualised returns of all three funds are shown
above. All three funds tend to follow each other and show significant
downturns during the GFC and Covid-19. Comparatively, the active
managers perform the worst, as for large periods of time there rolling
returns are substantially lower than the other two funds. The AI fund
seems capable of outperforming the benchmark, but never for long periods
of time.

\begin{figure}[H]

{\centering \includegraphics{Question1_files/figure-latex/Figure4-1} 

}

\caption{Density of Returns by Funds}\label{fig:Figure4}
\end{figure}

The density functions of returns, grouped by fund, show a clear result.
Active managers tend to perform below the market benchmark. The marker
benchmark here was defined as the average return of the benchmark. The
market benchmark and the AI fund perform very similar. The key
difference here lies in the flatter distribution of the AI fund, meaning
it is less consistent that the market benchmark.

\begin{figure}[H]

{\centering \includegraphics{Question1_files/figure-latex/Figure5-1} 

}

\caption{Comparison of Best and Worst Active funds vs AI Fund}\label{fig:Figure5}
\end{figure}

The figure reveals an interesting story. It appears that the AI fund
often lies between the rolling returns of these two funds.
Interestingly, for a period in the middle of our sample, the worst
performing fund, actually outperformed the best performing fund. This
potentially speaks to the shortcomings of average returns as a metric of
success. In this period, the AI fund also outperformed both of these
fund. The AI fund performs lower than the worst fund at times, but this
usually does not last long.

\bibliography{Tex/ref}





\end{document}
